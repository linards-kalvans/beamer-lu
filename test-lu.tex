% Par pamatu izmantota KDE beamer sagatave http://www.kde.org/kdeslides/
\documentclass[12pt]{beamer}

\usetheme{lu} % izsaucam LU stilu
\usepackage{amsmath} % matemātikai ..
\usepackage[no-math]{fontspec} % sistēmas fontiem ...
\usepackage{xunicode} % unikodam
\usepackage{xltxtra} % unikodam
\usepackage{pgf,pgfarrows,pgfnodes,pgfautomata,pgfheaps,pgfshade} % attēlu veidošanai

% Valodu atbalsts
%\usepackage{polyglossia} % var izmantot šo, ja nav fixlatvian
\usepackage{fixlatvian} % http://code.google.com/p/fixlatvian/
\setdefaultlanguage{latvian}
\setromanfont{Cambria}
\setsansfont{Calibri}
\setmonofont{Consolas}
% Virsraksts
\title[Dr. LaTeX Beamer]{Dr. \LaTeX~Beamer}
% Paplašināts virsraksts
\subtitle{Kā es iemācījos neuztraukties un man iepatikās slīdi}
%Autors
\author[T. Texmaker]{Tex Texmaker}
%Organizācija
\institute[LU FMF]{Latvijas Universitātes Fizikas un matemātikas fakultāte}
%Datums
\date{\today}

\begin{document}

\begin{frame}
  \titlepage
\end{frame}

\begin{frame}
  \frametitle{Saturs}
  \tableofcontents
\end{frame}

\AtBeginSection[]
{
  \frame<handout:0>
  {
    \frametitle{Saturs}
    \tableofcontents[currentsection,hideallsubsections]
  }
}

\AtBeginSubsection[]
{
  \frame<handout:0>
  {
    \frametitle{Saturs}
    \tableofcontents[sectionstyle=show/hide,subsectionstyle=show/shaded/hide]
  }
}

\newcommand<>{\highlighton}[1]{%
  \alt#2{\structure{#1}}{{#1}}
}

\newcommand{\icon}[1]{\pgfimage[height=1em]{#1}}



%%%%%%%%%%%%%%%%%%%%%%%%%%%%%%%%%%%%%%%%%
%%%%%%%%%% Šeit sākas saturs %%%%%%%%%%%%
%%%%%%%%%%%%%%%%%%%%%%%%%%%%%%%%%%%%%%%%%



\section{Ievads}

\begin{frame}
  \frametitle{Priekšnoteikumi \& mērķi}
  \framesubtitle{Nekas nav tik postošs kā nezināšanas un entuziasma duets}
  \begin{block}{LaTeX}
  \begin{itemize}
    \item Pirmkārt nepieciešamas kaut nelielas LaTeX priekšzināšanas;
    \item Šo to var iemācīties šeit.
  \end{itemize}
  \end{block}

  \begin{block}{Beamer}
  \begin{itemize}
    \item Tās vari apgūt aplūkojot šīs prezentācijas pirmkodu.
  \end{itemize}
  \end{block}

  \begin{block}{Mērķi}
  \begin{itemize}
    \item Iemācīties veidot strukturētus slīdus;
    \item Glītu tēmu lietošana;
    \item Iekarot visu pasauli;
    \item Atpūsties...
  \end{itemize}
  \end{block}
\end{frame}

\section{Pamatstruktūra}
\begin{frame}
	\frametitle{Nodaļas, slīdi un bloki}
	\framesubtitle{Sakārto visu ``kastītēs''}

	Šo nodaļu sauc "Pamatstruktūra". Un šis slīds ir tas, ko tagad redzi uz ekrāna.

	\begin{block}{Glīts bloks}
		Blokam ir nosaukums un saturs. Blokā vari ievietot praktiski visu, ko piedāvā LaTeX, piemēram, matemātiskās izteiksmes var ievietot, kā parasti:
		\begin{equation}
			\sum_{i=1}^n i = \frac{n \times (n+1)}{2}
		\end{equation}
	\end{block}

	Strādā arī ārpus bloka:
	\begin{equation}
		\sum_{i=1}^n i^2 = \frac{n \times (n+1) \times (2n+1)}{6}
	\end{equation}
\end{frame}

\begin{frame}
	\frametitle{Dažādi bloku tipi}
	\framesubtitle{Fui! Krāsas!!}
	\begin{block}{Standarta bloks}
		\begin{itemize}
			\item Standarta bloks, izmantojams grupēšanai.
			\item Protams, var saturēt sarakstus...
			\begin{itemize}
				\item Dažādos līmeņos...
				\item protams!
			\end{itemize}
		\end{itemize}
	\end{block}
	\begin{alertblock}{Trauksmes bloks}
		UZMANĪBU: Satur ļoti vērtīgu informāciju!
	\end{alertblock}
	\begin{example}
		Piemērus attēlo īpašā blokā...
	\end{example}
\end{frame}

\section{Foršas lietiņas}
\begin{frame}
	\frametitle{Teksta izcelšana un sadalīšana kolonnās}
	\framesubtitle{Hei! Skaties te!}
	\begin{columns}
		\column{.6\paperwidth}
			\begin{block}{Parasts bloks}
				\begin{itemize}
					\item Normāls teksts.
					\item \highlighton{Izcelts teksts}, lai pievērstu uzmanību .
					\item \alert{``Trauksmes'' teksts}, lai atzīmētu īpaši svarīgu informāciju.
					\item Kā alternatīvu var izmantot
					\begin{itemize}
						\alert{\item ``Trauksmi'' saraksta punktā}
						\highlighton{\item ``Izcelt'' punktu}
					\end{itemize}
				\end{itemize}
			\end{block}
		\column{.3\textwidth}
			\begin{alertblock}{Ja ir kas patiešām svarīgs, ...}
				\alert{... tad var veidot ``trauksmi'' ``trauksmes blokā''}\\
				Ewww, riebīgi, ne?
			\end{alertblock}
	\end{columns}
\end{frame}

\newcommand{\putlink}[1]{%
   \pgfsetlinewidth{1.4pt}%
   \pgfsetendarrow{\pgfarrowtriangle{4pt}}%
   \pgfline{\pgfxy(1,1)}{\pgfxy(#1,1)}
}

\begin{frame}
  \frametitle{Priekškara efekts}
  \framesubtitle{Saglabājam neziņu!}
  \begin{block}{Bumba ar laika degli}
  \begin{enumerate}
    \item<2-> Divi ...
    \item<3-> Viens ...
    \item<4-> Pēdejā iespēja ...
    \item<5-> BUMS!
  \end{enumerate}
  \end{block}
  \begin{block}{``Animācija"}<6->
    \begin{pgfpicture}{0cm}{0cm}{7cm}{2cm}
    \only<1-6>{
      \putlink{2}
    }
    \only<7>{
      \putlink{4}
    }
    \only<8>{
      \putlink{6}
    }
    \only<9>{
      \putlink{8}
    }
    \only<10>{
      \putlink{10}
    }
    \end{pgfpicture}
  \end{block}
\end{frame}

\section*{}
\frame{
  \vfill
  \centering
  \highlighton{
  \usebeamerfont*{frametitle}Un nu?

  \usebeamerfont*{framesubtitle}Apskatīsim slepeno nodaļu
  }
  \vfill
}
\begin{frame}
  \frametitle{Dod savu ieguldījumiu}
  \framesubtitle{Mums tevi vajag!}

  \begin{block}{Kāpēc?}
  \begin{itemize}
    \item \textit{Bīmeris} ir foršs!
    \item Šo tēmu vēl var uzlabot
  \end{itemize}
  \end{block}

  \begin{block}{Kā?}
  \begin{itemize}
    \item Ņem to!
    \item Uzlabo tās LaTeX kodu!
    \item Izmanto savu radošo potenciālu!
    \item Dokumentē!
    \item Palīdzi citiem cilvēkiem to izmantot!
    \item Lieto ...
  \end{itemize}
  \end{block}
\end{frame}

\begin{frame}
  \frametitle{Resursi}
  \framesubtitle{Ja vēlies šo tēmu uzlabot}
  \begin{thebibliography}{10}

  \beamertemplatearticlebibitems

  \bibitem{beamer-homepage}
    LaTeX Beamer
    \newblock {\tt http://latex-beamer.sourceforge.net/}

  \bibitem{beamer-themes}
    LaTeX Beamer Theme Matrix
    \newblock {\tt http://www.hartwork.org/beamer-theme-matrix/}

  \bibitem{kdeslides}
    KDE Presentations
    \newblock {\tt http://www.kde.org/kdeslides/}

  \end{thebibliography}
\end{frame}

\frame{
  \vspace{2cm}
  {\huge Jautājumi?}

  \vspace{3cm}
  \begin{flushright}
    Tex Texmaker

    \structure{\footnotesize{Paldies!}}
  \end{flushright}
}

\end{document}
